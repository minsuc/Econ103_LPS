%%%%%%%%%%%%%%%%% PREAMBLE %%%%%%%%%%%%%%%%%%%%%%%%%%%%
%Change the font size of your document - 10pt, 12.1pt, etc.
\documentclass[letterpaper,11pt,oneside]{article}
\usepackage[utf8]{inputenc}
\usepackage{setspace}
\usepackage{hyperref}

\usepackage{graphicx}
\graphicspath{ {images/}} %upload your signature to this file
%Change the margins to fit your CV/resume content
\usepackage[left=0.7in, right=0.7in, bottom=1.25in, top=1.25in]{geometry}

%Skype information - include your Skype name for a link to add you on Skype
\newcommand*{\Skype}{\href{skype:minsu.chang1111?add}{minsu.chang1111}} 
\newcommand{\Absender}[1][\normalsize]{\Skype} 

%Changes the page numbers - {arabic}=arabic numerals, {gobble}=no page numbers, {roman}=Roman numerals
\pagenumbering{gobble}

%%%%%%%%%%%%%%%%% END OF PREAMBLE %%%%%%%%%%%%%%%%%%%%%

\begin{document}

%%%%%%%%%%%%%%%%% NAME OF APPLICANT %%%%%%%%%%%%%%%%%%%

\noindent  \LARGE{\textbf{Minsu Chang}}  \\
\vspace{-1ex}
\hrule
\normalsize

%%%%%%%%%%%%%%%%% CONTACT INFORMATION %%%%%%%%%%%%%%%%%
% Your email address, website, and Skype name are links to send email, open your website and add you on Skype. 

\begin{center}
\begin{tabular}{l l}
 University of Pennsylvania    & \hspace{1in} \href{mailto:minsuc@sas.upenn.edu}{minsuc@sas.upenn.edu} \\
 Department of Economics    & \hspace{1in}  \href{www.minsuchang.com}{www.minsuchang.com}   \\
 3718 Locust Walk, Philadelphia, 19104              & \hspace{1in} Office: McNeil 363 \\%\Absender  \\
\end{tabular}
\end{center}

\vspace{1em}

%%%%%%%%%%%%%%%%% MAIN BODY %%%%%%%%%%%%%%%%%%%%%%%%%%%
% The main body is contained in a tabular environment. To move sections onto the next page, simply end the tabular environment and begin a new tabular environment.

\noindent \begin{tabular}{@{} l l}
 \large{\textbf{Education}} 
    & \textbf{University of Pennsylvania (Penn)} \\
     & Ph.D., Economics, degree expected May 2019. \\
     & Fields: Bayesian econometrics, Household finance, Computational macroeconomics \\
     & M.A., Economics, May 2016. \\
     & \\
     & \textbf{Seoul National Univeristy (SNU)} \\
          & M.A., Economics, Aug 2013. \\
     & B.A., \textit{Summa Cum Laude}, Economics, Aug 2011. \\
     \vspace{0.1in} \\
      \large{\textbf{Publications}}    & \textbf{``A Generalized Focused Information Criterion for GMM with } \\
 &\textbf{Applications to Panel Data Models,"}  \textit{Journal of Applied Econometrics},\\
 & 2018, 33 (3), pp.378-397 (with Francis J. DiTraglia) \\
    & \\
 & \textbf{``Nonparametric Tests of Conditional Treatment Effects with}\\
 & \textbf{an Application to Single-Sex Schooling on Academic Achievements,"} \\
 & \textit{The Econometrics Journal}, 2015, 18 (3), pp.307-346 (with Sokbae S. Lee and \\
 & Yoon-Jae Whang) \\ 
  \vspace{0.1in} \\
 \large{\textbf{Working papers}}    & \textbf{``Marriage, Home Ownership and Portfolio Share"} (Job Market Paper)\\
 & --- \textit{Abstract coming soon!}\\[0.1in]
    & \\
  & \textbf{``Does Heterogeneity Matter for Aggregate Fluctuations?"}\\
  & (with Xiaohong Chen and Frank Schorfheide)\\[0.05in] 
  & We specify a vector autoregressive model that stacks macroeconomic aggregates\\
 & and cross-sectional distributions to provide semi-structural evidence about the \\
 & interaction of aggregate and distributional dynamics. The specification of our\\
 & functional VAR is motivated by a linearization of a reduced-form model where\\
 & dynamics of aggregates and a function of lagged cross-sectional distribution of\\ & individual-level decisions or states, and the units (households or firms) base their\\ & decisions on lagged macroeconomic aggregates and cross-sectional distributions.\\ & To make the functional VAR analysis tractable, we approximate the log-densities\\ & of the cross-sectional distributions as well as the transition kernels in the functional\\ 
 & VARs by sieves. We apply our techniques to study the dynamics of technology\\ & shocks, per capita GDP, employment, and the earnings distribution.\\[0.1in]
   \end{tabular}\\
\noindent \begin{tabular}{@{} l l}
 & \textbf{``Lifting the Curse of Dimensionality With a Mixture Approximation:}\\
 &\textbf{Feasible Multivariate Density Estimation"} (with Paul Sangrey)\\[0.05in]
 & The curse of dimensionality makes multivariate nonparametric density estimation\\
 & infeasible with more than a few series. We provide Bayesian nonparametric density\\
 & estimators whose complexity is independent of the number of series by exploiting\\
 & the concentration of probability measures in high dimensions. We study Gaussian\\
 & hidden Markov processes providing a mixture representation where the number of\\
 & components grows logarithmically with the time dimension $T$. We consider an\\
 & asymptotic experiment where the time series dimension $T$ grows and the number\\
 & of series is fixed. Our marginal density estimator contracts at a rate $\sqrt{\log(T)}/\sqrt{T}$,\\ & and our transition density estimator contracts at a rate $\log(T)/\sqrt{T}$. We provide a \\ & computationally efficient Bayesian estimator using Dirichlet processes and apply it \\ & to two empirical examples.\\[0.1in]
    & \\
 \large{\textbf{Work}}  & \textbf{``Estimating Heterogeneous Agent Models:
A Likelihood Approach}\\
 \large{\textbf{in progress}}& \textbf{with Particle Filter"}\\[0.05in]
& Heterogeneous agent macroeconomic models have mostly relied on calibration,\\
&which cannot make testable statements regarding
parameter uncertainty and its \\
&statistical significance. This paper suggests how to estimate a
heterogeneous agent\\
& model similar to Krusell and Smith (1998) with Bayesian
particle filtering.\\
     \vspace{0.1in} \\
  \large{\textbf{Teaching}}    &  Instructor, PIER Workshop on Quantitative Tools for Macroeconomics, Penn, 2018\\ 
  \large{\textbf{experience}} &   Instructor, Statistics for Economists (Undergraduate), Penn, Summer 2017\\
& Instructor, PIER Workshop on Quantitative Tools for Macroeconomics, Penn, 2017\\% Policy Analysis \\
      & Teaching Assistant, Econometrics I (Ph.D.), Penn, Fall 2014/Fall 2015 \\
     & Teaching Assistant, Econometrics (Undergraduate), Penn, Spring 2015\\
     & Teaching Assistant, Econometrics (Undergraduate), SNU, Fall 2011/Spring 2012\\
     \vspace{0.1in} \\
       \large{\textbf{Other}}   & Dissertation Fellow, Federal Reserve Bank of Richmond, Summer 2016\\
     \large{\textbf{experience}}  & Research Assistant to Professor Frank Schorfheide (Penn), Jan 2016 - Present \\
     & Research Assistant to Professor Frank Schorfheide (Penn), July 2015 - Aug 2015\\
     & Research Assistant to Professor Yoon-Jae Whang (SNU), Aug 2011 - Aug 2013\\
     \vspace{0.1in} \\
 \large\textbf{{Awards and \quad}}   & Maloof Family Dissertation Fellowship, University of Pennsylvania, 2017-2018   \\
  \large{\textbf{fellowships}}   & University Fellowship, University of Pennsylvania, 2013-2014  \\
  & Kwanjeong Educational Foundation Scholarship, 2013-2017\\
     & BK21 Scholarship, National Research Foundation of Korea, 2011-2012\\
     & Graduate of Highest Honor, Seoul National University, Aug 2011 \\
     \vspace{0.1in} \\
 \large\textbf{{Referee \quad}}   & International  Economic Review\\
     \vspace{0.1in} \\
  \large{\textbf{Languages}}   & English (fluent), Korean (native), French (reading knowledge) \\
\large{\textbf{and skills}}    & Julia, MATLAB, Python, R (Rcpp), Stata, \LaTeX \\
\end{tabular}


\end{document}


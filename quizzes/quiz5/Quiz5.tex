\documentclass[addpoints,10pt]{exam}
\usepackage{amsmath, amssymb}
\usepackage{ifthen}
\usepackage{graphicx}
\usepackage{multirow}
\usepackage{units}
\usepackage[top=0.6in, left=0.6in, right=0.6in, bottom=0.6in]{geometry}
\usepackage{enumerate} %allows lettered lists in enumerate

%%%%%%%%%%%%%%%%%%%%%%%%%%%%%%%%%%%%%%%
%-------------SHOWTO COMMAND-------------
%%%%%%%%%%%%%%%%%%%%%%%%%%%%%%%%%%%%%%%

%This is a convenience command for conditional compilation that controls which questions are shown to a particular recitation section on the quiz. Any question not enclosed by \showto is shown to all recitation sections. The first argument to \showto is the desired section number. The second argument is the question body
\newcommand{\showto}[2]
{
\ifthenelse{
	 \equal{\RecitationSection}{#1}\OR\equal{\RecitationSection}{MASTER}
	}{#2}{}
}
%%%%%%%%%%%%%%%%%%%%%%%%%%%%%%%%%%%%%%%
%-----------END SHOWTO COMMAND------------
%%%%%%%%%%%%%%%%%%%%%%%%%%%%%%%%%%%%%%%



%%%%%%%%%%%%%%%%%%%%%%%%%%%%%%%%%%%%%%%
%------------------OPTIONS------------------
%%%%%%%%%%%%%%%%%%%%%%%%%%%%%%%%%%%%%%%
%\boxedpoints
%\pointsinmargin

%Show or hide answers
\printanswers
%\noprintanswers
%%%%%%%%%%%%%%%%%%%%%%%%%%%%%%%%%%%%%%%
%---------------END OPTIONS-----------------
%%%%%%%%%%%%%%%%%%%%%%%%%%%%%%%%%%%%%%%



%%%%%%%%%%%%%%%%%%%%%%%%%%%%%%%%%%%%%%%
%------------ SELECT SECTION-----------------
%%%%%%%%%%%%%%%%%%%%%%%%%%%%%%%%%%%%%%%

%To switch between different versions of the quiz for different recitation sections, change the final argument as needed to 201, 202, 203, 204, etc. If you instead enter MASTER you'll get a document with all quiz questions.
\newcommand{\RecitationSection}{205}

%%%%%%%%%%%%%%%%%%%%%%%%%%%%%%%%%%%%%%%
%------------END SELECT SECTION--------------
%%%%%%%%%%%%%%%%%%%%%%%%%%%%%%%%%%%%%%%



%%%%%%%%%%%%%%%%%%%%%%%%%%%%%%%%%%%%%%%
%---------------QUIZ HEADER-----------------
%%%%%%%%%%%%%%%%%%%%%%%%%%%%%%%%%%%%%%%
\begin{document}
\thispagestyle{empty}

\noindent \textbf{Econ 103 -- Quiz 5}

\vspace{15pt}
\noindent
\makebox[0.45\textwidth]{Name:\enspace\hrulefill}\hfill 
\paragraph{Instructions: } This is closed-book, closed-notes quiz. Please write your answers in the blanks provided. Non-programmable calculators are permitted.

\vspace{5pt}
%%%%%%%%%%%%%%%%%%%%%%%%%%%%%%%%%%%%%%%
%-------------END QUIZ HEADER----------------
%%%%%%%%%%%%%%%%%%%%%%%%%%%%%%%%%%%%%%%

\setlength\answerlinelength{6in}

\begin{questions}

\showto{205}{\question (2 points) Suppose you have $W_1,...W_{12} \sim iid\, N(3,1)$   independently of $V_1,...V_{6}\sim iid\, N(2,1)$. \\[0.1in] The random variable $\frac{\left(\bar W_{12}-\bar V_{6}-(3-2) \right)}{\sqrt{\frac{1}{12}+\frac{1}{6}}}$ follows a: 
\\
a. Standard Normal distribution\\[0.1in]
b. Normal distribution $N(\mu=1,\sigma^2=1)$\\[0.1in]
c. $\chi^2$ distribution\\[0.1in]
d. F-distribution
\vspace{0.35in}
\answerline[]}

%%%%%%%%%%%%%%%% Question 3


\showto{205}{\question (2 points) The results of a recent survey suggests that 25\% of Americans  are able to hold a conversation in a second language. For this survey 1,000 people were polled. Use the Central Limit Theorem to construct an approximate 95\% confidence interval for the actual proportion of Americans that are able to hold a conversation in a second language. \\ \\ \\ \\ 
\vspace{1in}
\answerline[]}


%%%%%%%%%%%%%%%% Question 4
%
%\showto{201}{\question The Central Limit Theorem states that if $X_1,...X_n$ are a random sample from a population with unknown mean $\mu$ then, for large $n$ the distribution of $\frac{\bar X_n-\mu}{\hat{SE}(\bar X_n)}$ is approximately a standard normal distribution. True or false?
%
%\answerline[True]}
%
%\showto{202}{\question The Central Limit Theorem states that if $X_1,...X_n$ are a random sample from a population with unknown mean $\mu$ then, for $n$ large $\bar X_n$ follows a standard normal distribution . True or False?
%\answerline[False]}
%
%\showto{203}{\question The Central Limit Theorem states that if $X_1,...X_n$ are a random sample from a population with unknown mean $\mu$ then, for $n$ large $\bar X_n$ is a consistent estimator of $\mu$?
%\answerline[False]}
%
%\showto{204}{\question The Central Limit Theorem states that if $X_1,...X_n$ are a random sample from a population with unknown mean $\mu$ then, for $n$ large $\frac{\bar X_n-\mu}{\hat{SE}(\bar X_n)}$ Is an unbiased estimator of $\mu$  True or false?
%\answerline[False]}
%%%%%%%%%%%%%%%% Question 5


%\showto{201}{\question True or false: Suppose we have two individual samples $X_1,...,X_n \sim iid$ with mean $\mu_X$ and variance $\sigma^2_X$ and $Y_1,...,Y_m \sim iid$ with mean $\mu_Y$ and variance $\sigma^2_Y$, where $corr(X,Y) = 0.5$. Then a 68\% confidence interval of the difference of population means is given by $(\bar{X}_n-\bar{Y}_m) \pm \sqrt{\frac{S^2_x}{n} + \frac{S^2_y}{m}} $
%\vspace{0.3in}
%\answerline[]}


%%%%%%%%%%%%%%%% Question 6
%\showto{204}{\question True or false: Suppose we take the mean of a random sample. The 95\% confidence interval for the population mean is roughly $\frac{1}{2}$ as wide as the 68\% confidence interval \answerline[]}


%%%%%%%%%%%%%%%% Question 7 & 8
% Table generated by Excel2LaTeX from sheet 'Sheet1'
%\begin{table}[htbp]
 % \centering
 % \caption{Fictional summary statistics for midterms this semester}
 %   \begin{tabular}{rrrr}
   % \toprule
  %      & \textbf{Midterm 1} & \textbf{Midterm 2} & \textbf{Difference} \\
   % \midrule
  %  Sample size & 100 & 100 & 100 \\
   % Sample mean & 110 & 100 & 10 \\
    %Sample standard deviation & 20  & 25  &  \\
  %  Sample variance & 400 & 625 &  \\
   % Sample correlation & \multicolumn{2}{c}{0.5} &  \\
   % \bottomrule
   % \end{tabular}%
%  \label{tab:addlabel}%
%\end{table}%


%\showto{201}{\question Calculate the variance of differences between midterm 1 and midterm 2 scores \\ \\ \answerline[]}

%\showto{201}{\question Suppose we want to know whether midterm 2 was `harder' than midterm 1. Using your answer to the previous question, and the numbers in the table, write down an expression for the 95\% confidence interval for the difference between scores on the midterm. (Note - you can leave your answer in terms of a square root) \\ \\ \answerline[]}

%%%%%%%%%%%%%%%% Question 2

\showto{205}{\question (1 point) True or false: a Type I error is when you fail to reject a null that is false. \answerline[]}



%%%%%%%%%%%%%%%% Question 4


\showto{205}{\question (2 points) Steve wants to test the null hypothesis that freshman students weigh on average 170 pounds against the two-sided alternative. Write down the null hypothesis and alternative hypothesis, where $\mu$ is the true average weight in the population.\\
\vspace{1in}
\answerline[]}

%%%%%%%%%%%%%%%% Question 5


\showto{205}{\question (2 points) Following on from the previous question, to test this hypothesis Steve gathered data on 9 students. We assume that the weights collected come from a normal distribution: $X_1...X_9\sim iid N(\mu,\sigma^2)$. He calculates the sample mean weight, $\bar{X}$. We assume that he knows the variance $\sigma^2$. Write down an expression for an appropriate test statistic  if he wants to test the null hypothesis $\mu=\mu_0$.
\answerline[]  
}


%%%%%% Question 6 %%%%%%%


\showto{205}{\question (2 points) Suppose you are testing $\mu=\mu_0$ when $X_1,...X_n \sim N(\mu,\sigma^2)$. Which of the following will be our rejection criterion if we are testing against the two sided alternative hypothesis $\mu\neq \mu_0$ with a significance level of $\alpha$?
\begin{enumerate}
	\item $\left| \frac{\bar X_n-\mu_0}{\sigma/\sqrt{n}}\right|>\mbox{qnorm}(1-\alpha/2)$. 
	\item $\frac{\bar X_n-\mu_0}{\sigma/\sqrt{n}}>\mbox{qnorm}(\alpha)$. 
	\item $\frac{\bar X_n-\mu_0}{\sigma/\sqrt{n}}<\mbox{qnorm}(1-\alpha)$. 
	\item $\frac{\bar X_n-\mu_0}{\sigma/\sqrt{n}}>\mbox{qnorm}(\alpha/2)$. 
\end{enumerate}
\answerline[]  
}

%%%%%% Question 7 %%%%%%%


\showto{205}{\question (2 points) Which of the following is true about $p$ values?
\\[0.1in]
a. $p$ value gives the probability that the null hypothesis is true.\\[0.1in]
b. $p$ value is the probability under the null hypothesis of observing a test statistic at least as extreme as the one actually obtained.\\[0.1in]
c. $p$ value gives us reliable results in the sense that, if we could repeat the experiment a great number of times, we would obtain a significant results on $100p$\% of those occasions. \\[0.1in]
d. $p$ value greater than the significance level leads to rejection of the null hypothesis.\\[0.1in]
\answerline[]}


%%%%%%%%%%%%%%%%% Questions 5, 6


\showto{205}{\question (3 points) From a polling data, 45\% of 130 Republicans correctly knew that John Roberts is the current Chief Justice whereas 40\% of 120 Democrats knew it. We can model each sample as draws from i.i.d. Bernoulli distribution. Suppose we want to test whether the population parameters from the two samples are equal. Calculate the value of the test statistic \underline{fully imposing} the null hypothesis. 
\vspace{2.3in} \answerline[]}


\showto{205}{\question (2 points) Suppose we want to conduct hypothesis testing against two-sided alternative with 5\% significance level. Given the test statistic you computed above, express the p-value using R command.
\\
\vspace{0.2in}\answerline[]}

\showto{205}{\question (2 points) Following on from the previous question, should we reject the null hypothesis? Why?
\\
\vspace{0.2in}\answerline[]}



%%%%%%%%%%%%%%%
\end{questions}

\end{document}  
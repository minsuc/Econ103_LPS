\documentclass[addpoints,10pt]{exam}
\usepackage{amsmath, amssymb}
\usepackage{ifthen}
\usepackage{graphicx}
\usepackage{multirow}
\usepackage{units}
\usepackage[top=0.6in, left=0.6in, right=0.6in, bottom=0.6in]{geometry}


%%%%%%%%%%%%%%%%%%%%%%%%%%%%%%%%%%%%%%%
%-------------SHOWTO COMMAND-------------
%%%%%%%%%%%%%%%%%%%%%%%%%%%%%%%%%%%%%%%

%This is a convenience command for conditional compilation that controls which questions are shown to a particular recitation section on the quiz. Any question not enclosed by \showto is shown to all recitation sections. The first argument to \showto is the desired section number. The second argument is the question body
\newcommand{\showto}[2]
{
\ifthenelse{
	 \equal{\RecitationSection}{#1}\OR\equal{\RecitationSection}{MASTER}
	}{#2}{}
}
%%%%%%%%%%%%%%%%%%%%%%%%%%%%%%%%%%%%%%%
%-----------END SHOWTO COMMAND------------
%%%%%%%%%%%%%%%%%%%%%%%%%%%%%%%%%%%%%%%



%%%%%%%%%%%%%%%%%%%%%%%%%%%%%%%%%%%%%%%
%------------------OPTIONS------------------
%%%%%%%%%%%%%%%%%%%%%%%%%%%%%%%%%%%%%%%
%\boxedpoints
%\pointsinmargin

%Show or hide answers
%\printanswers
\noprintanswers
%%%%%%%%%%%%%%%%%%%%%%%%%%%%%%%%%%%%%%%
%---------------END OPTIONS-----------------
%%%%%%%%%%%%%%%%%%%%%%%%%%%%%%%%%%%%%%%





%%%%%%%%%%%%%%%%%%%%%%%%%%%%%%%%%%%%%%%
%------------ SELECT SECTION-----------------
%%%%%%%%%%%%%%%%%%%%%%%%%%%%%%%%%%%%%%%

%To switch between different versions of the quiz for different recitation sections, change the final argument as needed to 201, 202, 203, 204, etc. If you instead enter MASTER you'll get a document with all quiz questions.
\newcommand{\RecitationSection}{201}

%%%%%%%%%%%%%%%%%%%%%%%%%%%%%%%%%%%%%%%
%------------END SELECT SECTION--------------
%%%%%%%%%%%%%%%%%%%%%%%%%%%%%%%%%%%%%%%





%%%%%%%%%%%%%%%%%%%%%%%%%%%%%%%%%%%%%%%
%---------------QUIZ HEADER-----------------
%%%%%%%%%%%%%%%%%%%%%%%%%%%%%%%%%%%%%%%
\begin{document}
\thispagestyle{empty}

\noindent \textbf{Econ 103 -- Quiz 3}

\vspace{15pt}
\noindent
\makebox[0.45\textwidth]{Name:\enspace\hrulefill}\hfill 
\paragraph{Instructions: } This is closed-book, closed-notes quiz. Please write your answers in the blanks provided. Each question is worth one point but no partial credit will be awarded. Non-programmable calculators are permitted.

\vspace{5pt}
%%%%%%%%%%%%%%%%%%%%%%%%%%%%%%%%%%%%%%%
%-------------END QUIZ HEADER----------------
%%%%%%%%%%%%%%%%%%%%%%%%%%%%%%%%%%%%%%%

\setlength\answerlinelength{2in}

\begin{questions}




%%%%%%%%%%%%%%%% Question 3
\showto{201}{\question A continuous random variable $X$ has a probability density function $f(x)$. It is possible that for some value we have $f(x)>1$. True or false? \answerline[True]}


\showto{202}{\question A continuous random variable $X$ has a probability density function $f(x)$. We interpret $f(3)$ as the probability that the random variable takes the value of 3. True or false? \answerline[False]}


\showto{203}{\question  A continuous random variable $X$ has a probability density function $f(x)$. It is possible to have $\int_{-\infty}^{\infty}f(x)>1$. True or false? \answerline[False]}

\showto{204}{\question A continuous random variable $X$ has a probability density function $f(x)$. We interpret $f(3)$ as the probability that the random variable takes the value of 3. True or false? \answerline[False]}

\showto{205}{\question  A continuous random variable $X$ has a probability density function $f(x)$. It is possible that for some value $f(x)<0$ for some $x$. True or false? \answerline[False]}

%%%%%%%%%%%%%%%%%%% Question 4

\showto{201}{\question The random variable $X$ follows a Uniform(0,1) distribution. Write down its probability density function, $f(x)$ as a piecewise function (Hint - make sure you consider all possible cases for $x$) 
 \answerline[]}

\showto{201}{\question The random variable $X$ follows a Uniform(0,1) distribution. Write down its CDF, $F(x_0)$ as a piecewise function (Hint - make sure you consider all possible cases for $x_0$) \vspace{0.5in}\answerline []}



\showto{201}{\question The random variable $X$ follows a Uniform(0,1) distribution. What is $P(0.3 \le X \le 0.5)$? \answerline[$0.2$]}

\showto{201}{\question Suppose $X \sim N(\mu, \sigma^2)$. What is $P(\mu - 2\sigma \leq X \leq \mu +2\sigma)$? \answerline[$0.95$]}

\showto{201}{\question Suppose $X_1, X_2, X_3 \sim i.i.d. N(\mu, \sigma^2).$ Let $\bar{X} = (X_1+X_2+X_3)/3.$ What is the distribution of $\bar{X}$? \answerline[]}

\showto{201}{\question Suppose $X_1, X_2, \ldots, X_5 \sim i.i.d. N(0,  1).$ Let $W = X_1^2+ X_2^2 + \ldots + X_5^2.$ What is the distribution of $W$? \answerline[]}

\showto{201}{\question Suppose $X_1, X_2, X_3 \sim i.i.d. N(0,  1).$ Let $Y = X_3/ \sqrt{(X_1^2 + X_2^2)/2}.$ What is the distribution of $Y$? \answerline[]}




\end{questions}

\end{document}  
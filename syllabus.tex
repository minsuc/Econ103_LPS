\documentclass[11pt, letterpaper]{article}
\usepackage{geometry}
\geometry{margin=1in} 
\usepackage{setspace}
\linespread{1}
\usepackage{hyperref}
\usepackage{enumerate}
\usepackage{fancybox}
\usepackage{amsmath, amssymb}




%%% HEADERS & FOOTERS
\usepackage{fancyhdr} 
\pagestyle{fancy} % options: empty , plain , fancy
\renewcommand{\headrulewidth}{0.4pt} % customise the layout...
\rhead{\footnotesize Chang -- Summer 2017}
\lhead{\footnotesize Econ 103 Syllabus}
\renewcommand\footrulewidth{0pt}


\begin{document}


\thispagestyle{plain}

\begin{center}
\Large
\sc
\textbf{ECON 103 - Statistics for Economists}\\
\large
\textbf{Summer 2017}
\end{center}

\normalsize
\bigskip

\begin{center}
\small
GENERAL INFORMATION
\end{center}

\noindent \textbf{Instructor:} Minsu Chang 

\medskip

\noindent \textbf{Class Times:} MTWRF 9:00-10:45 am, McNeil room 103

\medskip


\noindent \textbf{Office Hours:} These will be two days a week, 10:45-11:45 am, McNeil room 363.


\medskip

 
\noindent \textbf{Course Website:} \url{http://ditraglia.com/Econ103Public} At this url you can view the semester calendar and download all lecture slides, problem sets, etc.
You can view your grades and log-on to the course discussion forum, Piazza, at \url{https://canvas.upenn.edu}.

\medskip

\noindent \textbf{Email:} minsuc@sas.upenn.edu


\vspace{0.2in}

\begin{center}
\small
COURSE DESCRIPTION AND PREREQUISITES
\end{center}


\noindent \textbf{Course Description:} 
This course will teach you how to learn from data and understand uncertainty using the ideas of probability theory and statistics. 
After completing this course you will be able to carry out simple statistical analyses of your own using the computer package R.


\medskip

%\noindent \textbf{Lecture Recordings:} 
%Audio and screen captures of the lectures for this course will be recorded and posted on \href{http://upenn.instructure.com}{Canvas}. 
%This is a great way to get caught up if you miss a lecture.
%Be aware, however, that the lecture recording system is not 100\% reliable and the instructors have no direct control over it: in a typical semester one or two lectures will fail to be recorded due to technical problems.
%If you encounter any problems with the recordings, please contact \texttt{mms-help@sas.upenn.edu} directly.
%\medskip

\noindent \textbf{Prerequisites:} The prerequisite for this course is multivariate calculus (Math 104 followed by
114 or 115). To do well in this course you will need to be comfortable with algebra, manipulating
sums, differentiation and partial differentiation, solving unconstrained optimization problems, and
integration. 
%To help you gauge your level of mathematical preparation, I will administer a short
%math diagnostic quiz early in the Course. This will not count towards your grade. It is primarily
%intended to indicate to you where your mathematical strengths and weaknesses are, and where you
%need to brush up on your math to do well in this course.


\vspace{0.2in}

\begin{center}
\small
TEXTBOOK AND SOFTWARE
\end{center}

\noindent \textbf{Required Text:} 
The textbook for this course is \emph{Introductory Statistics for Business and Economics}, 4\textsuperscript{th} Edition by Thomas H.\ and Ronald J.\ Wonnacott (WW4). This book is ancient, so cheap used copies are plentiful. While I suggest that you complete
the assigned readings, my lecture slides, which will be posted online at the start of each week, are
the final authority on course material. In particular, you are not responsible for material in the
textbook unless it is also covered in lecture, but you are responsible for material from lecture even
if it is not covered in the textbook.

\medskip


%\noindent \textbf{Required Technology: } 
%We will be using the ``clickers'' for experiments and class participation exercises during the semester.
%Both the ResponseCard NXT and the ResponseCard QT are fully compatible with the exercises we will complete this semester.
%Other ResponseCard models will only work for some of the exercises so to make sure you get full participation credit (see below) make sure you have the right model.
%You can buy or rent a clicker from the Penn Bookstore.
%As clicker participation will make up 5\% of your course grade it is important that you bring your clicker to each lecture. 
%I fully understand, however, that things can go wrong: your clicker might stop working or you might forget to bring it after pulling an all-nighter.
%For this reason you will \emph{automatically} be excused from clicker participation for four lectures: there is no need to inform us in advance or after the fact. There will, however, be no further exceptions. 
%I will begin keeping track of clicker participation in our second lecture. For more details, see ``Participation'' below. 
%Because clickers will determine a portion of your grade, their use is subject to the code of academic integrity, as explained below under ``Academic Integrity.'' 
%


\noindent \textbf{Required Software:} 
We will use the statistical package R via a front-end called RStudio throughout the course. 
Both R and RStudio are free and open source. Installation instructions appear on the last page of this syllabus.
RStudio is also available in the Undergraduate Data Analysis Lab (UDAL) in McNeil rooms 104 and 108--9. 
You will be taught to use R in lecture and through as series of tutorials that I will assign as homework. (See ``Homework'' below.)  
Additional R resources are listed on the last page of this syllabus.

\medskip

\noindent \textbf{Recommended Texts:} 
I will not use these explicitly, but in the past instructors have recommended two
supplementary texts for students who feel they may need extra help with the course material. First is the \emph{Student Workbook to accompany Introductory Statistics for Business and Economics 4\textsuperscript{th} Edition}. 
Used copies are available on \href{http://www.amazon.com/gp/offer-listing/0471508993/sr=/qid=/ref=olp_page_2?ie=UTF8&colid=&coliid=&condition=all&me=&qid=&shipPromoFilter=0&sort=sip&sr=&startIndex=10}{Amazon}. 
The workbook contains fully worked out solutions to all odd-numbered problems from the textbook along with additional practice problems and solutions.
If you're having trouble with R and prefer a printed book to the free online resources listed below, I suggest consulting \emph{The R Student Companion} by Brian Dennis.


\newpage

%\noindent \textbf{Departmental Course Policies: } 
%All Economics Department course policies are in force in Econ 103 even if not explicitly listed on this syllabus. 
%See: \url{http://economics.sas.upenn.edu/undergraduate-program/course-information/guidelines/policies} for full details. 

%\noindent \textbf{Academic Integrity: } 
%All suspected violations of the code of academic integrity as set forth in the Pennbook will be reported to the Office of Student Conduct. 
%Confirmed violations will result in a failing grade for the course. 
%We will check identification cards at exams so please to bring yours.


\begin{center}
\small
GRADING
\end{center}

\noindent \textbf{Two Options:}
	\begin{equation*}
	\begin{split}
		\mbox{Overall Score} = (20\% \times \mbox{Quizzes})  + (40\% \times \mbox{Midterm}) + (40\% \times \mbox{Final})
	\end{split}
	\end{equation*}

OR

	\begin{equation*}
	\begin{split}
		\mbox{Overall Score} =
(10\% \times \mbox{Participation})  + 
		 (10\% \times \mbox{Homeworks})  \\
		 + (20\% \times \mbox{Quizzes}) 
		 + (30\% \times \mbox{Midterm}) + (30\% \times \mbox{Final})
	\end{split}
	\end{equation*}
\medskip

After the first week of class, I will ask each of you whether or not you want to opt out of your
participation and homework contributing towards your final grade. In the past, we have had some
students state that they would prefer to be judged only on their performance in tests, as their
learning style does not favor daily homework assignments and regular participation. However,
I encourage you to include your homework and participation grades in your final grade for two
reasons. First, these categories will be the easiest ones in which to score close to 100\%. Second, to
do well in ECON 103 requires consistent work every day, to make sure you are keeping up with the
material and are practicing solving questions. By including participation and homeworks in your
final grade, you are committing yourself to keeping up with the material. %I will not curve grades in this class.


\medskip

%\noindent \textbf{Piazza:} 
%We will be using an online discussion forum called Piazza, accessible via \href{http://upenn.instructure.com}{Canvas}, for all written communication in this course.
%We will use Piazza to make course announcements, answer questions about course material and respond to private messages from individual students regarding personal issues.
%By asking your question and getting an answer on Piazza, you create a positive externality: other students benefit from your questions and you benefit from theirs.
%You can even post anonymously if asking questions publicly makes you uncomfortable. I will actively moderate Piazza both to answer questions and approve
%(or correct) answers written by your fellow-students. Posting on Piazza will count towards your participation score (an optional 10\% of the overall grade). See “Participation” below for more details.

%\medskip

\noindent \textbf{Participation:} Each lecture will feature activities in which you can earn participation credit
through in-class discussion. You will earn further participation credit based on the frequency and
quality of your contributions on Piazza, including questions, answers, and follow-ups. If your
attendance is high, and you participate actively in class and online, you will receive 100\%: these
are essentially “free points.” However, spamming Piazza with unhelpful contributions will not gain
you credit. Simply reading posts on Piazza is not sufficient to earn participation points: you must
contribute through questions, answers, and follow-ups.

\medskip

\noindent \textbf{Homeworks and R tutorials:} There will be a series of short daily homeworks, to make sure
you are keeping up with the material and know how to solve problems. I will set a few questions
related to the material covered each day (except for Thursday, when you should prepare for the
quiz on Friday). We will go through some of these the next day in class, after the lecture. These
will be posted on Piazza (with solutions) as the class progresses, and will include problems from
the textbook as well as some additional questions.\\
\indent There will also be weekly R Tutorials (with full solutions and code) posted on Piazza. R material will account for approximately 25\% of the points available on each midterm and final, as well
as occasional questions on the quizzes. It is therefore very important that you keep up with the R material by completing these tutorials, though I will not collect these assignments.\\
\indent You can opt for your homeworks (excluding R tutorials) to count for 10\% of your final grade,
and if you do, you should submit your answers to questions from that week each Thursday. I will
not be able to grade these all in detail, so as long as you have made good, honest attempts to solve
the majority of questions, you will receive full marks on the homeworks. Regardless of whether you
opt for these to count towards your grade, you should aim to complete the homework every day.
You may work in groups for these problems, if this helps you. If you work in groups and use the solution keys, do so responsibly: you gain nothing by simply
copying others or reading the solutions before you have made a good attempt at the problem.\\

\medskip



\noindent \textbf{Quizzes:} There will be 5 weekly quizzes, administered at the beginning of class every Friday. Each
quiz will cover basic material from the classes that week, since the last quiz. When calculating your
quiz average, I will drop your lowest score and weight the remaining quizzes evenly. No makeup
quizzes will be given. Quizzes will not be returned and answers will not be posted but I will go
over the solutions in class.

\medskip

\noindent \textbf{Exams:} There will be one 90-minute in-class midterm and a 90 minute final in the last class. Dates
for these can be found in the course calendar below. The midterm and final are weighted equally
for your final grade. The midterm covers material up to Lecture 9, but the final is comprehensive:
it will focus on the second half of the course but include several questions on earlier material.
To give you a sense of the style and level of difficulty to expect, I have posted all of my past exams with full solutions on the \href{http://ditraglia.com/Econ103Public}{course website}. Attendance at
all exams is mandatory. In exceptional circumstances, e.g. a death in the family or a documented illness, please contact me in advance, using a private message in Piazza or Canvas. Exam regrade
requests must be made within three days of receiving your graded exam. As I re-grade the entire
exam, your score could rise or fall. Exams will be photocopied before being returned and you may
write in pencil or pen. Scientific calculators are permitted, graphing calculators are not.

\medskip

\begin{center}
\small
TENTATIVE SCHEDULE
\end{center}

\begin{table}[htbp]\centering
\small
\begin{tabular}{l c c c c c c}
\hline
\hline
Day & Date & Lecture & OH & Quiz & Exam & R tutorial \\
\hline
Mon & 22-May & 1: Descriptive Stats \& Graphics I & & & &\\
Tue & 23-May & 2: Description States \& Graphics II & & & & Try R\\
Wed & 24-May & 3: Regression I & Yes & & &\\
Thu & 25-May & 4: Basic Probability I &  Yes & && 1\\
Fri & 26-May & 5: Basic Probability II & & Quiz 1 (L1-4) & & \\
\hline
Mon & 29-May & 6: Basic Probability III  & & &\\
Tue & 30-May & No Class (Memorial Day)\\
Wed & 31-May & 7: Discrete Random Variables I & & & \\
Thu & 1-Jun & 8: Discrete Random Variables II & Yes& &\\
Fri & 2-Jun & 9: Discrete Random Variables III & Yes& Quiz 2 (L5-8)& &2\\
\hline
Mon & 5-Jun & & & & Midterm &\\
Tue& 6-Jun & 10: Continuous Random Variables I &&&\\
Wed & 7-Jun & 11: Continuous Random Variables II & Yes&&\\
Thu & 8-Jun & 12: Sampling Distributions I & Yes &&&\\
Fri & 9-Jun & 13: Sampling Distributions II & & Quiz 3 (L9-L12) &  &3\\
\hline
Mon & 12-Jun & 14: Confidence Intervals I &&&&\\
Tue & 13-Jun & 15: Confidence Intervals II &&&&\\
Wed & 14-Jun & 16: Confidence Intervals III & Yes &&\\
Thu & 15-Jun & No Class &\\
Fri & 16-Jun & 17: Confidence Intervals IV & Yes & Quiz 4 (L13-16) & & 4 \\ 
\hline
Mon & 19-Jun & 18: Hypothesis Testing I & & &\\
Tue & 20-Jun & 19: Hypothesis Testing II &\\
Wed & 21-Jun & 20: Hypothesis Testing III& Yes\\
Thu & 22-Jun & 21: Hypothesis Testing IV& Yes\\
Fri & 23-Jun & 22: Regression II& & Quiz 5 (L17-21) &&5\\
\hline 
Mon & 26-Jun & 23: Regression III & Yes &\\
Tue & 27-Jun & 24: Review & Yes&\\
Wed & 28-Jun & && &Final &\\
\hline
\hline
\end{tabular}
%\caption{Summary Statistics}
\end{table}

\newpage

\begin{center}
\small
INSTALLING R AND RSTUDIO
\end{center}

First, download and install R from \url{http://cran.r-project.org/}. Second, download and install RStudio by visiting \url{http://rstudio.org/download/desktop} and clicking the link listed under ``Recommended for Your System.'' 
Here are links to some additional free resources to help you learn R:
\begin{itemize}
		       \item \url{http://cran.r-project.org/other-docs.html}
\item \url{http://www.twotorials.com/}
			\item \url{http://www.r-bloggers.com/google-developers-r-programming-video-lectures/}
		 	\item \url{http://cran.r-project.org/doc/contrib/Farnsworth-EconometricsInR.pdf}
 			\item \url{http://www.ats.ucla.edu/stat/R/}
\end{itemize}

\end{document}
